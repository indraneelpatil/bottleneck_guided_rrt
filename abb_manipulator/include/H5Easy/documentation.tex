\documentclass[11pt]{article}
\usepackage{graphicx}
\usepackage[margin=0.75in]{geometry}
\geometry{letterpaper}
\usepackage{amsmath}
\usepackage{listings}
\usepackage{color}
\usepackage{epstopdf}
\usepackage{color}
\graphicspath{ {Figures/} }
\DeclareGraphicsRule{.tif}{png}{.png}{`convert #1 `dirname #1`/`basename #1 .tif`.png}  %Need for pictures
\newcommand{\ignore}[1]{} %used to make inline comments
\definecolor{gray}{rgb}{0.1, 0.1, 0.1}
\newcommand{\gray}[1]{\colorbox{gray}{#1}}

\title{H5Easy documentation}
\date{\today}
\author{Steven Walton}

%\tcbset{
%       frame code={},
%       center title,
%       left=0pt,
%       right=0pt,
%       top=0pt,
%       bottom=0pt,
%       colback=gray!70,
%       colframe=white,
%       width=\dimexpr\textwidth\relax,
%       enlarge left by=0mm,
%       boxsep=5pt,
%       arc=0pt,outer arc=0pt,
%       }

\lstset{language=C++,
  basicstyle=\ttfamily,
  keywordstyle=\color{blue}\ttfamily,
  stringstyle=\color{red}\ttfamily,
  commentstyle=\color{green}\ttfamily,
  morecomment=[l][\color{magenta}]{\#}
}

\begin{document}
\maketitle
\tableofcontents
\section{Requirements}
Make sure that you have `libhdf5-dev' installed so that you have the correct headers. 
It is also suggested that you have `h5utils' and `hdf5-tools' installed.
\section{Usage}
The purpose of this library is to provide easy reading and writing to hdf5 format through
the use of C++ std::vector types. It will handle most standard types but currently only
writes to `STD\_U32LE', `STD\_I32LE', `IEEE\_F32LE', and `IEEE\_F64LE' (uint, int, float, 
and double) types. Reading will work for both big and little endian types as well
as different sized uint and int data. 
\\
NOTE that casting on the fly will not work. That is, you can not request double data 
by calling with a float.
\subsection{Compiling code}
To compile run the command
\begin{lstlisting}
   h5c++ -std=c++11 foo.cpp -o bar
\end{lstlisting}
We use auto and thus need the c++11 standard. \\
Note that if you have `Anaconda' installed that it comes with h5c++ but that it points 
to `/path/to/anaconda/lib' instead of the correct dev files that are installed when you
install `libhdf5-dev'. If you get undefined references check where `h5c++' is pointing.

\subsection{Write an H5 file}
To do this we use a class structure. To write a vector formatted data you would do
the following
%\begin{tcolorbox}
\begin{lstlisting}
   WriteH5 data;
   data.setFileName("myH5File.h5");
   data.setVarName("myVariableName");
   data.writeData(myVector);
\end{lstlisting}
%\end{tcolorbox}

You can also create groups

\begin{lstlisting}
   data.createGroup("/my/group/name");
   data.setVarName("/my/group/name/myVariableName");
   data.writeData(myVector);
\end{lstlisting}
With this you can also string together multiple commands.
\begin{lstlisting}
   WriteH5 data;
   data.setFileName("myH5File.h5");
   data.setVarName("var1");
   data.write(vector1);
   data.createGroup("/my/nested/group/");
   data.setVarName("/my/nested/group/var2");
   data.writeData(vector2);
   data.setVarName("/my/nested/group/var3");
   data.writeData(vector3);
\end{lstlisting}
On the back end we use a proxy array to store the data. This array is created on the 
heap to allow for bigger data to be written. This array is then deleted at the end
of the getData call.

\subsection{Reading H5 data}
To load the data we will need to know the type that the data is stored as. This is
where `h5dump -H' becomes helpful. You can inspect the data and see the type and size.
\begin{lstlisting}
   LoadH5 data;
   data.setFileName("myH5File.h5");
   data.setVarName("myVariableName");
   vector<type> loadedData = data.getData();
\end{lstlisting}
We can also handle groups similarly to above if we set the var name as the full path
name. We are also able to handle 2D vector data with the following
\begin{lstlisting}
   vector<vector<type> > md_vec = data.getData();
\end{lstlisting}
There is more functionality written into the load data than in the write data. The Load
class is able to read many different types of data. It inspects the data for the size 
and type. For example it will read both `IEEE\_F32LE' and `IEEE\_F32BE' data. Again
we are using a proxy array to read the data that is allocated to the heap. We also
use a proxy function to be able to return by the correct type. If you wish to add
new types to the read then you should edit both of these sections. 
\\
Do note that we can not cast on the fly. If the data is stored as `IEEE\_F64BE' then
you cannot call it with `vector<float>'. As of now you must call it as a double. This
is in the plans to be fixed.

\end{document}
